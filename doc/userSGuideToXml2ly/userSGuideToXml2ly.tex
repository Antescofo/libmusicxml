\documentclass[12pt,a4paper]{article}


% -------------------------------------------------------------------------
% import common LaTeX settings
% -------------------------------------------------------------------------

\usepackage{import}
\subimport{../}{CommonLaTeXSettings}


% -------------------------------------------------------------------------
\begin{document}
% -------------------------------------------------------------------------

%\noindent

\title{
User's guide to \xmlToLy\ \\[5pt]
}

\newsavebox{\authorBox}
\sbox{\authorBox}{
Former lecturer in computer science at Centre Universitaire d'Informatique, \newline
University of Geneva, Switzerland
}
%\usebox{\authorBox}

\author{
Jacques Menu 
%\footnote {
%Former lecturer in computer science at Centre Universitaire d'Informatique, 
%University of Geneva, Switzerland}
}

\date {\normalsize \today\ version}
%\date {}

\maketitle

\abstract {
This document presents the design principles behind \xmlToLy, as well as the way to use it. It is part of the \lib\ documentation, to be found at \url{https://github.com/grame-cncm/libmusicxml/tree/lilypond/doc}.

All the examples mentioned can be downloaded from \url{https://github.com/grame-cncm/libmusicxml/tree/lilypond/files/samples/musicxml}. They are grouped by subject in subdirectories, such as \mxmlfile{basic/HelloWorld.xml}.
}

% -------------------------------------------------------------------------
% -------------------------------------------------------------------------
\section{Acknowledgements}
% -------------------------------------------------------------------------
% -------------------------------------------------------------------------

Many thanks to Dominique Fober, the designeer and maintainer of the \lib\ library. This author would not have attempted to work on a \mxml\ to \lily\ translator without it already available.


% -------------------------------------------------------------------------
% -------------------------------------------------------------------------
\section{Overview of \xmlToLy\ }
% -------------------------------------------------------------------------
% -------------------------------------------------------------------------

% -------------------------------------------------------------------------
\subsection{Why \xmlToLy?}
% -------------------------------------------------------------------------

\lily\ comes with \mxmlToLy, a translator of \mxml\ files to \lily\ syntax, which has some limitations. Also, being written in Python, it is not in the main stream of the \lily\ development group. The latter has much to do with C++ and Scheme code already.

After looking at the \mxmlToLy\ source code, and not being a Python developper, this author decided to go for a new translator written in C++.

The desing goals for \xmlToLy\ were:
\begin{itemize}
\item to perform at least as well as \mxmlToLy;
\item to provide as many options as needed to adapt the \lcg\ to the user's needs.
\end{itemize}


% -------------------------------------------------------------------------
\subsection{What \xmlToLy\ does}
% -------------------------------------------------------------------------


% -------------------------------------------------------------------------
% postamble
% -------------------------------------------------------------------------

\pagebreak

\lstlistoflistings

%\listoffigures

\tableofcontents


% -------------------------------------------------------------------------
\end{document}
% -------------------------------------------------------------------------
